% !TEX root = ../FundationsDataScience.tex

\chapter{Convex Duality}
\label{chap-conv-duality}

Main ref \cite{chambolle2010introduction,chambolle2016introduction,boyd2004convex}, \cite{parikh2014proximal,boyd2011distributed,beck2014introduction}

TODO.







%%%%%%%%%%%%%%%%%%%%%%%%%%%%%%%%%%%%%%%%%%%%%%%%
\section{Forward-backward on the Dual}
\label{sec-fb-dual}

\paragraph{Chambolle's algorithm.}

Chambolle in \cite{chambolle-algo-tv}�detail an algorithm to minimize exactly the TV denoising problem
\eql{\label{eq-prox-tv}
	f_\la^\star = \uargmin{g \in \RR^N} \frac{1}{2} \norm{f-g}^2 + \la \normTV{g}.
}
It uses a relationship between the vectorial $\lun$ and $\linf$ norms
\eq{
	\normu{v} = \sum_{m=0}^{N-1} \norm{v_m} \qandq
	\normi{v} = \umax{0 \leq m < N} \norm{v_m}
}
where each $v_m \in \RR^2$ and $v \in \RR^{N \times 2}$. One has
\eq{
	\normu{v} = \umax{ \normi{w} \leq 1 } \dotp{w}{u}
}
which allows one to re-write the optimization \eqref{eq-prox-tv} as
\eq{
	\umin{g \in \RR^N} \umax{\normi{w} \leq 1} 
		\frac{1}{2} \norm{f-g}^2 + \la \dotp{w}{\nabla g}.
}
Exchanging the roles of the min and the max, one proves that the solution of  \eqref{eq-prox-tv} is re-written as
\eql{\label{eq-primal-dual-chambolle}
	f_\la^\star =  f + \la \diverg(w^\star)
}
where 
\eql{\label{eq-chambolle-dual}
	w^\star \in \uargmin{\normi{w} \leq 1} \norm{f + \la \diverg(w^\star)}^2.
}
The convex optimization problem \eqref{eq-chambolle-dual} computes a dual vector field $w^\star \in \RR^{N \times 2}$, from which the denoised image is recovered using \eqref{eq-primal-dual-chambolle}.

The dual problem \eqref{eq-chambolle-dual} is the minimization of a quadratic functional subject to a convex $\linf$ constraint. It can thus be solved using for instance a projected gradient descent
\eq{
	w^{(k+1)}_m = \frac{ \tilde w^{(k)}_m }{ \max(|\tilde w^{(k)}_m|,1) }
	\qwhereq
	\tilde w^{(k)} = w^{(k)} + \tau \nabla( f/\la + \diverg(w^{(k)}) ).
}  
If the gradient step size satisfy $0 < \tau < 1/4$, one can prove that 
\eq{
	f + \la \diverg(w^{(k)}) \longrightarrow f_\la^\star
	\qwhenq
	k \rightarrow +\infty.
}

