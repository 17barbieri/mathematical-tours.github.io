%!TEX root = ../FundationsDataScience.tex

\chapter{Machine Learning}


% Refs \cite{rosasco2017notes,friedman2001elements,murphy2012machine}

%%%%%%%%%%%%%%%%%%%%%%%%%%%%%%%%%%%%%%%%%%%%%%%%%%
%%%%%%%%%%%%%%%%%%%%%%%%%%%%%%%%%%%%%%%%%%%%%%%%%%
%%%%%%%%%%%%%%%%%%%%%%%%%%%%%%%%%%%%%%%%%%%%%%%%%%
\section{Supervised vs Unsupervised Learning}

TOOD: describe the general settings and problems. 

%%%%%%%%%%%%%%%%%%%%%%%%%%%%%%%%%%%%%%%%%%%%%%%%%%
%%%%%%%%%%%%%%%%%%%%%%%%%%%%%%%%%%%%%%%%%%%%%%%%%%
%%%%%%%%%%%%%%%%%%%%%%%%%%%%%%%%%%%%%%%%%%%%%%%%%%
\section{PCA, Nearest-Neighbors Classification and Clustering}


%%%%%%%%%%%%%%%%%%%%%%%%%%%%%%%%%%%%%%%%%%%%%%%%%%
\subsection{Dimenionality Reduction and PCA}

We detail Principal Component Analysis (dimentionality reduction), supervised classification using nearest neighbors
and unsupervised clustering using $k$-means.


We use here the famous
% <https://en.wikipedia.org/wiki/Iris_flower_data_set  
IRIS dataset of Fisher. 
The data set consists of 50 samples from each of three species of Iris (Iris setosa, Iris virginica and Iris versicolor).
 Four features were measured from each sample: the length and the width of the sepals and petals, in centimetres. 


The feature are stored as the row of a matrix $X \in \RR^{n \times p}$

$n$ is the number of samples, $p$ is the dimensionality of the features,
$k$ is the number of classes.


In order to display in 2-D or 3-D the data, dimensionality reduction is needed.
The simplest method is the Principal Component Analysis (PCA), 
which perform an orthogonal linear projection on the principal axis (eigenvector) of the
covariance matrix.

We compute the empirical mean
\eq{
    m = \frac{1}{n} \sum_{i=1}^n x_i \in \RR^p
} 
and  covariance
\eq{
 C = \frac{1}{n} \sum_{i=1}^n (x_i-m) (x_i-m)^\top \in \RR^{p \times p}. 
}
Denoting $\tilde X = X - 1_p m^\top $, one has $C=\tilde X^\top
\tilde X$. 

% saveas(gcf,[rep 'cov.png']);


We compute PCA ortho-basis using the SVD decomposition
\eq{
 \tilde X = U \diag(d) V  
}
where $U \in \RR^{n \times p}$ and $V \in \RR^{p \times p}$ have
orthonormal columns. $V$ are the principal directions of variance and
are order by decreasing variances. 

We then compute the feature in the PCA basis, $z_i=V^\top (x_i-m) $, stored in matrix format as $Z=\tilde X V$.


One can plot the singular values of the covariances, which corresponds to the standard deviation of the data 
along the principal directions.

% saveas(gcf,[rep 'svd.png']);


The first dimensions of the $z_i$ are the optimal way to linearly
embed the data in a low dimensional space. 
This can be used for display in 2-D using the first two dimension.
One can do a similar display in 3-D.

% saveas(gcf,[rep 'points-2d.png']);
% saveas(gcf,[rep 'points-3d.png']);


%%%%%%%%%%%%%%%%%%%%%%%%%%%%%%%%%%%%%%%%%%%%%%%%%%%%%%%%%%%%%%%%%%%%%%%%%%%%%%%%%%%%%%%%%%%%%%%%%%%%%%%%%%%%
\subsection{Supervised Learning: Nearest Neighbor Classification}

Probably the simplest method for supervised classification is Nearest
Neighbor ($R$-NN), where $R$ is a parameter indexing the number of neighbor.
Increasing $R$ is important to cope with noise and obtain smoother
decision boundary, and hence better generalization performance.

The class predicted for a point $x$ is the one which is the most
represented among the $R$ points $(x_i)_i$ which are the closed to
$x$.


The data needs to be split into training and testing.

One then compute Euclidean distance between some $x$ 
and all other $x_{1,j}$ in the training set. 
%
Then sort the distance and generate the list of sorted classes $ y_\sigma = (y_{\si(i)})_i$. This
generate an indexing $\si$ (a permutation of $\{1,\ldots,n\}$) such that 
\eq{
 \norm{x-x_{\si(1)}} \leq \norm{x-x_{\si(2)}} \leq \ldots \leq \norm{x-x_{\si(n)}}. 
}


For a given $R$, one can compute the histogram of class apparition
\eq{
 h_\ell \eqdef \frac{1}{R} \enscond{ i }{ \si(i) \in \{1,\ldots,R\} }
  = \sharp \si^{-1}( \{1,\ldots,R\} ). 
}
The decision class for $x$ is then the maximum of the histogram
\eq{
 c(x) \eqdef \uargmax{\ell} h_\ell 
}

One can display the histogram $(h_\ell)_\ell$ of repartition of class indexes as $R$ grows.
% 
Then perform the NN classification for all the points in the test set, and for varying $R$.
And show how the classification score $S$ (number of correctly classified points)
evolves with $R$.

% saveas(gcf,[rep 'hist-classif.eps'], 'epsc');
% saveas(gcf,[rep 'hist-classif-score.eps'], 'epsc');


One can also display, as a function of the position in 2-D PCA space, the class output by 
the $R$-NN method when applied in 2-D.


% saveas(gcf,[rep 'classif-vizu.eps'], 'epsc');


%%%%%%%%%%%%%%%%%%%%%%%%%%%%%%%%%%%%%%%%%%%%%%%%%%%%%%%%%%%%%%%%%%
\subsection{Unsupervised Learning: $k$-means}

In an un-supervised setting, the class information $y$ is not
available. The basic problem is then to recover class information from
the knowledge of $x$ only. This corresponds to the clustering problem.



The most basic algorithm is the $k$-means, which tries to recover the
class index $\bar y_i=\ell$ from the distance $ \norm{x_i-c_\ell} $
between the feature point $x_i$ and the class centroid $c_\ell$
(which are the unknown of the problem). 

It does so by minimizing the following non-convex energy
\eq{
 \umin{ (c_\ell)_\ell } \sum_i \umin{\ell} \norm{x_i-c_\ell}^2   
}
We first initialize the class centroids $ (c_\ell)_\ell $ at random among the points.
They are stored in as the row of a matrix $ C \in \RR^{k \times p} $.


The $k$-means algorithm iterate between first determining the class of
each point using the distance to the centroids
\eq{
 \forall i \in \{1,\ldots,n\}, \quad \bar y_i \leftarrow
          \uargmin{\ell} \norm{x_i-c_\ell}. 
}

One can display the centroids and the classes using colors.
This corresponds to a Voronoi diagram segmentation in the high
dimensional space, but here the display is done in 2D.

The second step of the $k$-means algorithm is to update the centroids
position to be the mean of the points inside each class
 \eq{
 \forall \ell \in \{1,\ldots,k\}, \quad c_\ell \leftarrow
      \frac{ \sum_{i:y_i=\ell} x_i }{ \sharp \{i:y_i=\ell\} }. 
}
One can then perform several step of the $k$-means algorithm. 
And display the histogram of (true, i.e. according to $y$) class  inside
each estimated class (i.e. according to $\bar y$).


% saveas(gcf,[rep 'kmeans-iter.eps'], 'epsc');

% saveas(gcf,[rep 'kmeans-classif-scores.eps'], 'epsc');

It is possible to implement better initialization strategies such as farthest point sampling or $k$-means++.



%%%%%%%%%%%%%%%%%%%%%%%%%%%%%%%%%%%%%%%%%%%%%%%%%%
%%%%%%%%%%%%%%%%%%%%%%%%%%%%%%%%%%%%%%%%%%%%%%%%%%
%%%%%%%%%%%%%%%%%%%%%%%%%%%%%%%%%%%%%%%%%%%%%%%%%%
\section{Linear Regression and Kernel Methods}

We now study linear regression method, and its non-linear variant using kernelization.

%%%%%%%%%%%%%%%%%%%%%%%%%%%%%%%%%%%%%%%%%%%%%%%%%%
\subsection{Linear Regression}

We test the method on the Boston house prices dataset, consisting in $n=506$ samples with
features $x_i \in \RR^p$ in dimension $p=13$. The goal is to predict the price value
$y_i \in \RR$. 

We separate the features $X$ from the data $y$ to predict information.
$n$ is the number of samples, $p$ is the dimensionality of the features,

In order to display in 2-D or 3-D the data, dimensionality is needed.
The simplest method is the principal component analysis, which perform an
orthogonal linear projection on the principal axsis (eigenvector) of the
covariance matrix.

We look for a linear relationship $ y_i = \dotp{w}{x_i} $
written in matrix format $ y= X w $
where the rows of $X \in \RR^{n \times p}$ stores the features $x_i \in \RR^p$.

Since here $n>p$, this is an over-determined system, which can
solved in the least square sense
\eq{
	 \umin{ w }  \norm{Xw-y}^2, 
}
whose solution is given using the Moore-Penrose pseudo-inverse
\eq{
	 w = (X^\top X)^{-1} X^\top y.
}

Regularization is obtained by introducing a penalty. It is often called
ridge regression, and is defined as 
\eq{
	 \umin{ w }  \norm{Xw-y}^2 + \lambda \norm{w}^2 
}
where $\lambda>0$ is the regularization parameter.

The solution is given using the following equivalent formula
\eq{
	 w = (X^\top X + \lambda \text{Id}_p )^{-1} X^\top y, 
}
\eq{
	 w = X^\top ( XX^\top + \lambda \text{Id}_n)^{-1} y, 
}
When $p<n$ (which is the case here), the first formula should be
preferred. 

In contrast, when the dimensionality $p$ of the feature is very
large and there is little data, the second is faster. Furthermore, this
second expression is generalizable to Kernel Hilbert space setting,
corresponding possibly to $p=+\infty$ for some kernels. 


%%%%%%%%%%%%%%%%%%%%%%%%%%%%%%%%%%%%%%%%%%%%%%%%%%
\subsection{Kernelized Ridge Regression}

In order to perform non-linear and non-parametric regression, it is
possible to use  kernelization. It is non-parametric in the sense that the number of
parameter grows with the number $n$ of samples (while for the initial
linear  method, the number of parameter is $p$). This allows in particular to
generate estimator of arbitrary complexity. 

Given a kernel $ \kappa(x,z) \in \RR $ defined for $(x,z) \in \RR^p \times \RR^p$,
the kernelized method replace the linear approximation functional $f(x) =
\dotp{x}{w}$ by a sum of kernel centered on the samples 
\eq{
	 f_h(x) = \sum_{i=1}^n h_i k(x_i,x) 
}
where $h \in \RR^n$ is the unknown vector of weight to find. 

When using the linear kernel $\kappa(x,y)=\dotp{x}{y}$, one retrieves
the previously studied linear method. 

% saveas(gcf,[rep 'scatter-2d.png']);


The gaussian kernel is the most well known and used kernel
\eq{
	 \kappa(x,y) \eqdef e^{-\frac{\norm{x-y}^2}{2\sigma^2}} . 
}
The bandwidth parameter $\si>0$ is crucial and controls the locality of
the model. It is typically tuned through cross validation.  

Once evaluated on grid points, the kernel define a matrix
\eq{
	 K = (\kappa(x_i,x_j))_{i,j=1}^n \in \RR^{n \times n}.  
}

The weights $h \in \RR^n $ are solutions of
\eq{
	 \umin{h} \norm{Kh-y}^2 + \la \dotp{Kh}{h}  
}
and hence can be computed by solving a linear system
\eq{
	 h = (K+\la \text{Id}_n)^{-1} y  
}

%  saveas(gcf,[rep 'regul-evol.png']);






%%%%%%%%%%%%%%%%%%%%%%%%%%%%%%%%%%%%%%%%%%%%%%%%%%
%%%%%%%%%%%%%%%%%%%%%%%%%%%%%%%%%%%%%%%%%%%%%%%%%%
%%%%%%%%%%%%%%%%%%%%%%%%%%%%%%%%%%%%%%%%%%%%%%%%%%
\section{Logistic Classification}

We now detail the logistic classification method (for 2 classes and
multi-classes).

Note thatlLogisitic classification is actually called ``logistic
regression'' in the literature, but it is in fact a classification method.



%%%%%%%%%%%%%%%%%%%%%%%%%%%%%%%%%%%%%%%%%%%%%%%%%%
\subsection{Two Classes Logistic Classification}

Logistic classification is, with support vector machine (SVM), the baseline
method to perform classification. Its main advantage over SVM is that is
is a smooth minimization problem, and that it also output class
probabity, offering a probabilistic interpretation of the classification.

To understand the behavior of the method, we generate synthetic data
distributed according to a mixture of Gaussian with an overlap governed by an offset $\omega$.
 Here classes indexes are set to $y_i \in
\{-1,1\}$ to simplify the equations.

% saveas(gcf,[rep 'classes-2.eps'], 'epsc');

Logistic classification minimizes a logistic loss in place of the usual
$\ell^2$ loss for regression
\eq{
	 \umin{w} E(w) \eqdef \frac{1}{n} \sum_{i=1}^n L(\dotp{x_i}{w},y_i)  
}
where the logistic loss reads
\eq{
	 L( s,y ) \eqdef \log( 1+\exp(-sy) ) 
}
This corresponds to a smooth convex minimization. If $X$ is injective,
this is also strictly convex, hence it has a single global minimum.

One can compare the binary (ideal) 0-1 loss, the logistic loss and the hinge loss (the one used for SVM).

% saveas(gcf,[rep 'losses.eps'], 'epsc');

This can be interpreted as a maximum likelihood estimator when one
models the probability of  belonging to the two classes for sample $x_i$ as
\eq{
	 h(x_i) \eqdef (\th(x_i),1-\th(x_i)) \qwhereq
          \th(s) \eqdef \frac{e^{s}}{1+e^s} = (1+e^{-s})^{-1}  
}

Re-writting the energy to minimize
\eq{
	 E(w) = \Ll(X w,y) \qwhereq \Ll(s,y)= \frac{1}{n}  \sum_i L(s_i,y_i), 
}
its gradient reads
\eq{
	 \nabla E(w) = X^\top \nabla \Ll(X w,y)
      \qwhereq
      \nabla \Ll(s,y) = \frac{y}{n} \odot \th(-y \odot s),   
}
where $\odot$ is the pointwise multiplication operator, i.e. \texttt{.*} in
Matlab.


In order to improve performance, it is important (especially
in low dimension $p$) to add a constant bias term $w_{p+1} \in \RR$, and replace $\dotp{x_i}{w}$
by $ \dotp{x_i}{w} + w_{p+1} $.  This is equivalently achieved by
adding an extra $(p+1)^{\text{th}}$ dimension equal to 1 to each
$x_i$, which we do using a convenient macro.

With this added bias term, once $w_{\ell=0} \in \RR^{p+1}$ initialized
(for instance at $0_{p+1}$),

one step of gradient descent reads
 \eq{
	 w_{\ell+1} = w_\ell - \tau_\ell \nabla E(w_\ell). 
}

One can then implement a gradient descent
 \eq{
	 w_{\ell+1} = w_\ell - \tau_\ell \nabla E(w_\ell). 
}


One can display the data overlaid on top of the
classification probability, this highlight the
separating hyperplane $ \enscond{x}{\dotp{w}{x}=0} $.



% saveas(gcf,[rep 'classes-2-separation-influ.png']);

%%%%%%%%%%%%%%%%%%%%%%%%%%%%%%%%%%%%%%%%%%%%%%%%%%%%%%%%%%%%%%%%%%%%%%%%%%%%%
\subsection{Kernelized Logistic Classification}

Logistic classification tries to separate the classes using
a linear separating hyperplane $ \enscond{x}{\dotp{w}{x}=0}. $

In order to generate a non-linear descision boundary, one can replace the
parametric linear model by a non-linear non-parametric model, thanks to
kernelization. It is non-parametric in the sense that the number of
parameter grows with the number $n$ of sample (while for the basic
method, the number of parameter is $p$. This allows in particular to
generate decision boundary of arbitrary complexity.

The downside is that the numerical complexity of the method grows
(at least) quadratically with $n$.

The good news however is that thanks to the theory of reproducing kernel Hilbert spaces
(RKHS), one can still compute this non-linear decision
function using (almost) the same numerical algorithm.

Given a kernel $ \kappa(x,z) \in \RR $ defined for $(x,z) \in \RR^p$,
the kernelized method replace the linear decision functional $f(x) =
\dotp{x}{w}$ by a sum of kernel centered on the samples
\eq{
	 f_h(x) = \sum_{i=1}^p h_i k(x_i,x) 
}
where $h \in \RR^n$ is the unknown vector of weight to find.

When using the linear kernel $\kappa(x,y)=\dotp{x}{y}$, one retrieves
the previously studied linear method.


The gaussian kernel is the most well known and used kernel
\eq{
	 \kappa(x,y) \eqdef e^{-\frac{\norm{x-y}^2}{2\sigma^2}} . 
}
The bandwidth parameter $\si>0$ is crucial and controls the locality of
the model. It is typically tuned through cross validation.

%  saveas(gcf,[rep 'classes-kernel.eps'], 'epsc');

Once evaluated on grid points, the kernel define a matrix
\eq{
	 K = (\kappa(x_i,x_j))_{i,j=1}^n \in \RR^{n \times n}.  
}

Valid kernels are those that gives rise to positive symmetric matrices
$K$. The linear and Gaussian kernel are valid kernel functions. Other
popular kernels include the polynomial kernel $ \dotp{x}{y}^a $ for $a
\geq 1$ and the Laplacian kernel $ \exp( -\norm{x-y}^2/\si ) $.

The kernelized Logistic minimization reads
\eq{
	 \umin{h} F(h) \eqdef \Ll(K h,y). 
}

This minimization can be related to an infinite dimensional optimization
problem where one minimizes directly over the function $f$. This
is shown to be equivalent to the above finite-dimenisonal optimization problem
thanks to the theory of RKHS.
One can then use a gradient descent to minimize $F(h)$.


Once this optimal $h$ has been found, class probability at a point
$x$ are obtained as
\eq{
	 (\th(f_h(x)), 1-\th(f_h(x)) 
}
where $f_h$ has been defined above.


% saveas(gcf,[rep 'classes-kernel-result.png'], 'png');


One can separate the dataset into a training set and a testing set. Evaluate the classification performance
for varying $\si$. Try to introduce regularization and minmize
\eq{
	 \umin{h} F(h) \eqdef \Ll(K h,y) + \la R(h) 
}
where for instance $R=\norm{\cdot}_2^2$ or  $R=\norm{\cdot}_1$.


%%%%%%%%%%%%%%%%%%%%%%%%%%%%%%%%%%%%%%%%%%%%%%%%%%%%%%%%%%%%
\subsection{Multi-Classes Logistic Classification}

The logistic classification method is extended to an arbitrary number
$k$ of classes by considering a family of weight vectors $ ( w_\ell )_{\ell=1}^k $, which are conveniently stored as columns of matrix $W \in \RR^{p \times k}$.

This allows to model probabilistically the belonging of a point $x \in \RR^p $ to a
the classes using an exponential model
\eq{
	 h(x) = \pa{ \frac{ e^{-\dotp{x}{w_\ell}} }{ \sum_m e^{-\dotp{x}{w_m}} } }_\ell 
}
This vector $h(x) \in [0,1]^k $ describes the probability of $x$
belonging to the different classes, and $ \sum_\ell h(x)_\ell = 1 $.

The computation of $w$ is obtained by solving a maximum likelihood
estimator
 \eq{
	 \umax{w \in \RR^k} \frac{1}{n} \sum_{i=1}^n \log( h(x_i)_{y_i} ) 
}
where we recall that $y_i \in \{1,\ldots,k\}$ is the class index of
point $x_i$.

This is conveniently rewritten as
\eq{
	 \umin{w} \sum_i \text{LSE}( XW )_i - \dotp{XW}{D} 
}
where $D \in \{0,1\}^{n \times k}$ is the binary class index matrices
\eq{
	  D_{i,\ell} = \choice{
          1 \qifq y_i=\ell, \\
          0 \text{otherwise}.
      }
}
and LSE is the log-sum-exp operator
\eq{
	 \text{LSE}(S) = \log\pa{ \sum_\ell \exp(S_{i,\ell}) } \in \RR^n. 
}


The computation of LSE is
unstable for large value of $S_{i,\ell}$ (numerical overflow, producing NaN), but this can be
fixed by substracting the largest element in each row,
since $ \text{LSE}(S+a)=\text{LSE}(S)+a $ if $a$ is constant along rows. This is
the LSE trick.


The gradient of the LSE operator is the soft-max operator
\eq{
	  \nabla \text{LSE}(S) = \text{SM}(S) \eqdef
      \pa{
          \frac{
                  e^{S_{i,\ell}}
              }{
                  \sum_m e^{S_{i,m}}
              } }   
}
Similarly to the LSE, it needs to be stabilized.


We load a dataset of $n$ images of size $p = 8 \times 8$, representing digits from 0
to 9 (so there are $k=10$ classes).
%
Separate the features $X$ from the data $y$ to predict information.
%
$n$ is the number of samples, $p$ is the dimensionality of the features, $k$
the number of classes.
%
One can display a few samples digits


% saveas(gcf,[rep 'digits.png'], 'png');
% saveas(gcf,[rep 'digits-2d.png'], 'png');
% saveas(gcf,[rep 'digits-3d.png'], 'png');

We compute the $D$ matrix and define the energy $E(W)$.
%
Then define its gradients
\eq{
	 \nabla E(W) =  \frac{1}{n} X^\top ( \text{SM}(X W) - D ).  
}
and one can use a gradient descent
 \eq{
	 W_{\ell+1} = W_\ell - \tau_\ell \nabla E(W_\ell). 
}
We can evaluate class probability associated to weight vectors on this grid.


% saveas(gcf,[rep 'digits-classes-all.png'], 'png');
% saveas(gcf,[rep 'digits-classes-single.png'], 'png');
