% !TEX root = ../IntroImaging.tex

\chapter{Image Processing}
\label{chap-images}

\newcommand{\FootLink}[1]{\footnote{\url{#1}}}
\newcommand{\myparagraph}[1]{\vspace{2mm}\noindent\textbf{#1}}
\newcommand{\lien}[2]{#2}

%\newcommand{\bs}{\_}

\newcommand{\BetweenImageSpace}{\hspace{3mm}}

\newcommand{\image}[1]{\includegraphics[width=0.32\linewidth]{#1}}
\newcommand{\imageM}[1]{\includegraphics[width=0.4\linewidth]{#1}}
\newcommand{\imageL}[1]{\includegraphics[width=0.6\linewidth]{#1}}
\newcommand{\imageQuad}[8]{
\begin{tabular}{@{}c@{\BetweenImageSpace{}}c@{}}
\image{#1} & \image{#2}\\
#5 & #6 \\
\image{#3} & \image{#4}\\
#5 & #6 
\end{tabular}
}
\newcommand{\imageTri}[6]{
\begin{tabular}{@{}c@{\BetweenImageSpace{}}c@{\BetweenImageSpace{}}c@{}}
\image{#1} & \image{#2} & \image{#3}\\
#4 & #5 & #6 \\
\end{tabular}
}
\newcommand{\imageDuo}[4]{
\begin{tabular}{@{}c@{\BetweenImageSpace{}}c@{}}
\image{#1} & \image{#2} \\
#3 & #4  \\
\end{tabular}
}

\newcommand{\myfigure}[3]{
    \begin{figure}[ht]
    \begin{center}
        #1                          % usual include graphics
    \end{center}
    \vspace{-4mm}
        \caption{\textit{#2}}       % caption
    \label{#3}          % label
    \end{figure}
}



Digital cameras take precise photographs of the world around us. The user wants to be able to store his photos on his hard drive with minimum memory requirement. He also wishes to be able to reprocess them in order to improve their quality. This article presents the mathematical and computer tools used to perform these different tasks. 

%%%%%%%%%%%%%%%%%%%%%%%%%%%%%%%%%%%%%%%%%%%%%%%%%%%%%%%%%%%%%%
%%%%%%%%%%%%%%%%%%%%%%%%%%%%%%%%%%%%%%%%%%%%%%%%%%%%%%%%%%%%%%
%%%%%%%%%%%%%%%%%%%%%%%%%%%%%%%%%%%%%%%%%%%%%%%%%%%%%%%%%%%%%%
\section{The pixels of an image}

A \lien{https://en.wikipedia.org/wiki/Digital_image}{digital image}
in gray levels is an array of values. Each
box of this table, which stores a value, is called a
\lien{https://en.wikipedia.org/wiki/Pixel}{pixel}.
By noting $n$ the number of rows and $p$ the number of columns in the image,
we manipulate an array of $n \times p$ pixels. Figure \ref{fig-section1-original-zoom}, left, shows a visualization of a square table with $n = p = 240$, which represents $240\times 240$ = 57600 pixels. The
\lien{https://en.wikipedia.org/wiki/Digital_camera}{digital cameras}
can record much larger images,
with several millions of pixels.

The values of the pixels are stored in
\lien{http://en.wikipedia.org/wiki/Computer}{a computer} or
a digital camera in the form of
\lien{https://en.wikipedia.org/wiki/Integer}{relative integers} between entre 0 et $255=2^8-1$,
making 256 possible values for each pixel. The value 0 is black, and the value 255 is white. The intermediate values correspond to
\lien{https://en.wikipedia.org/wiki/Grayscale}{gray levels}
ranging from black to white. Figure \ref{fig-section1-original-zoom} shows a subset of $6 \times 6$ pixels taken from the previous image. You can see both the values that make up the table and the gray levels that allow you to display the image on the screen.

\myfigure{
\imageL{section1-original-zoom}
}{Sub-image of size $5 \times 5$}{fig-section1-original-zoom}

%%%%%%%%%%%%%%%%%%%%%%%%%%%%%%%%%%%%%%%%%%%%%%%%%% %%%%%%%%%%%%
%%%%%%%%%%%%%%%%%%%%%%%%%%%%%%%%%%%%%%%%%%%%%%%%%% %%%%%%%%%%%%
\section{Image Storage}

%%%%%%%%%%%%%%%%%%%%%%%%%%%%%%%%%%%%%%%%%%%%%%%%%% %%%%%%%%%%%%
\subsection{Binary Codes}

Storing large images on the
\lien{https://en.wikipedia.org/wiki/Hard_disk_drive}{hard drive}
of a computer takes
a significant amount of places. Integer numbers are stored
in \lien{https://en.wikipedia.org/wiki/Binary_number}{binary},
in the form of a succession
of 0 and 1. Each 0 and each 1 corresponds to an elementary unit
of information, called bit.
To obtain the binary expression of a pixel having the value 179,
it is necessary to decompose this value as a sum of powers of two.
We thus obtain
\[ 
	179=2^7+2^5+2^4+2+1, 
\]
where care has been taken to order the powers of two in decaying order. In order to make the binary more explicit,
we add ``$1 \times$" before each power that appears in the expression,
and ``$0\times$" before the powers that do not appear
\[ 179=1 \times 2^7 + 0 \times 2^6 + 1 \times 2^5 + 1 \times 2^4 + 
  0 \times 2^3 + 0 \times 2^2 + 1 \times 2^1 + 1 \times 2^0. \]
Using such a decomposition, the value of each pixel, which is a number between 0 and 255, requires
$\log_2(256) = 8\text{bits}$. 
The binary writing of the value 179 of the pixel is thus $(1,0,1,1,0,0,1,1)$.
Any value between 0 and 255 can be written in this way, which requires the use of 8 bits. Indeed,
there are 256 possible values, and $256 = 2^8$. To store the complete image, it is therefore necessary to use $n \times p \times 8 \text{bits}$.
For the image shown in the previous figures, it is thus necessary to use
\[
	256 \times 256 \times 8 = 524288 \text{bits}. 
\]
Equivalently, this image requires 57.6kb (kilobytes), since a kilobyte is equal to $8$ bits.


%%%%%%%%%%%%%%%%%%%%%%%%%%%%%%%%%%%%%%%%%%%%%%%%%% %%%%%%%%%%%%
\subsection{Sub-sampling an Image}


\myfigure{
\imageTri
%{section2-subsample-2}
{section2-subsample-4}
{section2-subsample-8}
{section2-subsample-16}
%{Une ligne/colonne sur 2}
{One row / column of 4}
{One row / column out of 8}
{One row / column of 16}
}{Subsampling of an image}{fig-section2-subsample}

In order to reduce the required storage space of an image, the number of pixels can be reduced.
The easiest way to do this is to delete rows and columns in the original image. Figure \ref{fig-section2-subsample}, at the top left, shows what is obtained if one row is kept out of 4 and one column out of 4. We thus have divided by $4 \times 4 = 16$ the number of pixels of the image, and thus also divided by 16 the number of bits required to store the image on
a hard disc. In fig. \ref{fig-section2-subsample}, one can see the results obtained by removing more and more rows and columns. Of course, the quality of the image degrades quickly.

%%%%%%%%%%%%%%%%%%%%%%%%%%%%%%%%%%%%%%%%%%%%%%%%%% %%%%%%%%%%%%
\subsection{Quantizing an image}

Another way to reduce the memory space required for storage
is to use fewer integers for each value. For example, we can use only integers between 0 and 3, which will give an image with only 4 levels of gray. One can convert the original image to an image with 4 levels of values by performing the replacements:
\begin{rs}
\item the values in $0,1,\ldots, 63$ are replaced by the value 0 (black),
\item the values in $64,1,\ldots, 127$ are replaced by the value 1 (light gray),
The values in $128.1,\ldots, 191$ are replaced by the value 2 (dark gray),
\item the values in $192,\ldots, 255$ are replaced by the value 3 (white).
\end{rs}
Such an operation is called \lien{https://en.wikipedia.org/wiki/Quantization}{quantization}. Figure~\ref{fig-section3-quantize}, in the center, shows the resulting image with 4 grayscale levels.

We have already seen that we can represent any value between 0 and 255 using 8 bits using binary coding. In the same way,
one can check that any value between 0 and 3 can be represented using 2 bits.
This thus results in a reduction of a factor 8/2=4 of the memory footprint necessary
for storing the image on a hard disk. Figure \ref{fig-section3-quantize} shows the results obtained using less and less gray levels.

\myfigure{
\imageDuo
%{section3-quantize-2}
%{section3-quantize-3}
{section3-quantize-4}
{section3-quantize-16}
%{16 niveaux de gris}
%{3 niveaux de gris}
{4 niveaux de gris}
{16 niveaux de gris}
}{Quantizing an image
}{fig-section3-quantize}
 
As with the reduction of the number of pixels, the reduction of the number
gray levels greatly affects the quality of the image.
In order to minimize the size of an image without changing its quality,
more complex methods of
\lien{https://en.wikipedia.org/wiki/Image_compression}{image compression}.
The most effective method is JPEG-2000. It uses the theory of wavelets.




%%%%%%%%%%%%%%%%%%%%%%%%%%%%%%%%%%%%%%%%%%%%%%%%%% %%%%%%%%%%%%
%%%%%%%%%%%%%%%%%%%%%%%%%%%%%%%%%%%%%%%%%%%%%%%%%% %%%%%%%%%%%%
%%%%%%%%%%%%%%%%%%%%%%%%%%%%%%%%%%%%%%%%%%%%%%%%%% %%%%%%%%%%%%
\section{Noise Removal}

%%%%%%%%%%%%%%%%%%%%%%%%%%%%%%%%%%%%%%%%%%%%%%%%%% %%%%%%%%%%%%
\subsection{Local Averaging}

Images are sometimes of poor quality. A typical example of a defect
is the noise which appears when a picture is under-exposed,
that is if there is not enough light. This noise corresponds to small
random fluctuations of the gray levels. Figure~\ref{fig-section5-moyenne}, on the left, shows such 
a noisy image.


\myfigure{
\imageM{section5-moyenne-numbers}
}{Pixels neighborhood.
}{fig-section5-moyenne-numbers}

In order to remove the noise in the images, it is necessary to
modify the pixel values.
The simplest operation is to replace the value
$a$ of each pixel by the average of
$a$ and the values $b, c, d, e, f, g, h, i$ of the 8 pixels that surround $a$. Figure \ref{fig-section5-moyenne-numbers} shows an example of a neighborhood of 9 pixels. A modified image is thus obtained by replacing $a$ by
\[\frac{a + b + c + d + e + f + g + h + i}{9}, \]
since the average of 9 values is averaged.
%
In our example, this average is
\[ \frac{190+192+79+54+47+153+203+189+166}{9} \approx 141. \]
By performing this operation for each pixel, a large part of 
the noise is removed, because this noise is made up of random fluctuations, which are
decreased by averaging. Figure \ref{fig-section5-moyenne}, top left, shows the effect of such a calculation. All the noise was not removed by this operation. In order to remove more
of noise, one can average more values around each pixel.
%
Figure \ref{fig-section5-moyenne} shows the result obtained by increasing the average
of values.


\myfigure{
\imageTri
{section5-moyenne-1}
{section5-moyenne-2}
{section5-moyenne-3}
%{section5-moyenne-4}
{Mean on 9 pixels}
{Mean on 25 pixels}
{Mean on 49 pixels}
%{Moyenne sur 81 pixels}
}{Mean with increasing width
}{fig-section5-moyenne}

Pixel averaging is very effective in removing noise in
images, unfortunately it also destroys much of the
the information of the image. It can indeed be seen that the images
obtained by averaging are blurry.
This is particularly visible near contours, which are not sharp.

%%%%%%%%%%%%%%%%%%%%%%%%%%%%%%%%%%%%%%%%%%%%%%%%%% %%%%%%%%%%%%
\subsection{Local Median}

To reduce this blur, the mean can be replaced by the median. In the example of the neighborhood of 9 pixels used in the previous section, the 9 sorted values are:
\[47, 54, 79, 153, 166, 189, 190, 192, 203. \]
The median of these nine values is 166. In order to remove more noise, it is enough to calculate the median over a larger number of neighboring pixels, as shown in Figure \ref{fig-section6-mediane}.
One can observe that this method is more efficient than the mean calculation 
because the resulting images are less blurry. However, just as
with the calculation of averages, if we take neighborhoods that are too large, we lose
also information of the image, especially the edges of the objects are degraded.


\myfigure{
\imageTri
{section6-mediane-1}
{section6-mediane-2}
{section6-mediane-3}
%{section6-mediane-4}
{Median on 9 pixels}
{Median on 25 pixels}
{Median on 49 pixels}
%{M�diane sur 81 pixels}
}{Median filtering with increasing windowing size.
}{fig-section6-mediane}




%%%%%%%%%%%%%%%%%%%%%%%%%%%%%%%%%%%%%%%%%%%%%%%%%% %%%%%%%%%%%%
%%%%%%%%%%%%%%%%%%%%%%%%%%%%%%%%%%%%%%%%%%%%%%%%%% %%%%%%%%%%%%
%%%%%%%%%%%%%%%%%%%%%%%%%%%%%%%%%%%%%%%%%%%%%%%%%% %%%%%%%%%%%%
\section{Detecting Edges of Objects}

In order to locate objects in the images, it is necessary to
detect their edges. These edges correspond to
areas of the image where pixel values change rapidly. It is the case
for example when passing from the boat (which is dark,
ie. with small values) to the sea (which is clear, therefore with
large values).

In order to know if a pixel with a value $a$ is along an edge of an object, the values $b, c, d, e$ of its four neighbors are taken into account, which have a common side with it (Figure \ref{fig-section7-contours-numbers}). This allows the edges of objects to be detected as accurately as possible.


\myfigure{
\imageM{section7-contours-numbers}
}{Example of a neighborhood of 5 pixels.
}{fig-section7-contours-numbers}


A value $\ell$ is calculated according to the formula
\[ \ell = \sqrt{ (b-d)^2 + (c-e)^2 }.  \]
In our example, we thus obtain
\[ \ell= \sqrt{�(192 - 153)^2 + (189 - 54)^2 } = \sqrt{19746} \approx 141. \]
It may be noted that if $\ell = 0$, then $b = c$
and $d = e$. On the contrary, if
$\ell$ is large, this means that the neighboring pixels have very high values
different, the pixel considered is therefore probably on the edge of an object.

Figure \ref{fig-section7-contours} shows an image whose pixel value is $\min(\ell, 255)$.
It is necessary to take the minium with 255, because the value of $\ell$ can exceed the maximum displayable value (255, which corresponds to white). These values are displayed with black when $\ell = 0$, in white when $\ell$ is high, and gray levels are used for the intermediate values. It can be seen that in the image on the right, the outlines of the objects appear white, as they correspond to  large values of $\ell$.

\myfigure{
\imageDuo
{section7-image}
{section7-contours}
{Original Image}
{Contour map $\ell$}
}{Edge detection.
}{fig-section7-contours}

%%%%%%%%%%%%%%%%%%%%%%%%%%%%%%%%%%%%%%%%%%%%%%%%%%%%%%%%%%%%%%
%%%%%%%%%%%%%%%%%%%%%%%%%%%%%%%%%%%%%%%%%%%%%%%%%%%%%%%%%%%%%%
%%%%%%%%%%%%%%%%%%%%%%%%%%%%%%%%%%%%%%%%%%%%%%%%%%%%%%%%%%%%%%
\section{Color Images}


%%%%%%%%%%%%%%%%%%%%%%%%%%%%%%%%%%%%%%%%%%%%%%%%%% %%%%%%%%%%%%
\subsection{RGB Space}

A
\lien{https://en.wikipedia.org/wiki/Color_space}{color image}
is actually composed of three independent images,
in order to represent the
\lien{https://en.wikipedia.org/wiki/RGB_color_model}{red, green, and blue}.
Each of these three
images is called a
\lien{https://en.wikipedia.org/wiki/Channel_(digital_image)}{color channel}.
This representation in red, green and blue mimics the
human visual system.
Figure~\ref{fig-canaux-coul} shows the three constituent channels of the image shown on the left of Figure~\ref{fig-luminance}.

\myfigure{
\imageDuo
{section8-image}
{section8-luminance}
{Original image}
{Luminance}
}{Color image.
}{fig-luminance}


Each pixel of the color image thus contains three numbers $(r, v, b)$,
each being an integer between 0 and 255.
If the pixel is equal to $(r, v, b) = (255,0,0)$, it contains only information
red, and is displayed as red.
Similarly, the pixels of $(0,255,0)$ and $(0,0,255)$ are
respectively displayed green and blue.


\myfigure{
\imageTri
{section8-rouge}
{section8-vert}
{section8-bleu}
{Red canal }
{Green canal}
{Blue canal}
}{Color channels
}{fig-canaux-coul}


A color image can be displayed on the screen.
from its three channels $(r, v, b)$ using the rules of
\lien{https://en.wikipedia.org/wiki/Additive_color}{additive color synthesis}. These rules correspond to the way in which light rays combine, hence the qualifier \guill{additive}.
Figure \ref{fig-section8-synthese}, left, shows the composition rules
this additive synthesis of colors.
For example, a pixel with the values
$(r, v, b) = (255,0,255)$ is a mixture of red and green,
displayed as yellow.


\myfigure{
\imageDuo
{section8-synthese-additive}
{section8-synthese-soustractive}
{Additive synthesis}
{Subtractive synthesis}
}{Color Synthesis
}{fig-section8-synthese}


%%%%%%%%%%%%%%%%%%%%%%%%%%%%%%%%%%%%%%%%%%%%%%%%%% %%%%%%%%%%%%
\subsection{CMJ Space}

Another common representation for color images uses
as background colors cyan, magenta and yellow. It is calculated
the three numbers $(c, m, j)$ corresponding to each of these three channels
from the red, green and blue channels $(r, v, b)$ as follows
\[c = 255-r, \quad m = 255-v, \quad j = 255-b. \]
For example, a pixel of pure blue
$(r, v, b) = (0,0,255)$ will become
$(c, m, j) = (255,255,0)$. Figure~\ref{fig-canaux-cmj} shows the three channels
$(c, m, j)$ of a color image.

\myfigure{
\imageTri
{section8-cyan}
{section8-magenta}
{section8-jaune}
{Cyan canal}
{Magenta canal}
{Yellow canal}
}{Channels CMJ
}{fig-canaux-cmj}


In order to display a color image on the screen from the three channels $(c, m, j)$, the \lien{https://en.wikipedia.org/wiki/Subtractive_color}{subtractive color synthesis} must be used. Figure \ref{fig-section8-synthese}, right, shows the composition rules of this subtractive synthesis. They correspond in painting to the absorption of light by colored pigments, hence the qualifier \guill{subtractive}. Cyan, magenta and yellow are called primary colors.

It is thus possible to store on a hard disk a color image by storing the
three channels, corresponding to the $(r, g, b)$ or $(c, m, j)$.
%
One can change color images in a similar way as graylevel image, by changing each channel.


%%%%%%%%%%%%%%%%%%%%%%%%%%%%%%%%%%%%%%%%%%%%%%%%%%%%%%%%%%%%%%
%%%%%%%%%%%%%%%%%%%%%%%%%%%%%%%%%%%%%%%%%%%%%%%%%%%%%%%%%%%%%%
%%%%%%%%%%%%%%%%%%%%%%%%%%%%%%%%%%%%%%%%%%%%%%%%%%%%%%%%%%%%%%
\section{Changing the Contrast of an Image}

%%%%%%%%%%%%%%%%%%%%%%%%%%%%%%%%%%%%%%%%%%%%%%%%%% %%%%%%%%%%%%
\subsection{Luminance}

One calculates a grayscale image from a color image
as the mean of the three channels. Thus, for each pixel, a value
\[a = \frac{r + v + b}{3} \]
is computed which is called luminance
of the color. Figure \ref{fig-luminance} shows the transition from a color image to a luminance image in grayscales. 


%%%%%%%%%%%%%%%%%%%%%%%%%%%%%%%%%%%%%%%%%%%%%%%%%% %%%%%%%%%%%%
\subsection{Grayscale contrast manipulations}

It is possible to make various changes to the image in order to
modify his contrast. We consider here a grayscale image.
A simple manipulation consists in replacing each value $a$ of a pixel of an image by $255-a$, which corresponds to the opposite gray intensity. The white becomes black and vice-et-versa, giving a similar effect to that
of the negative of film for cameras, see figure \ref{fig-section4-square}, left.

The image is lightened or darkened using an increasing function from $[0,255]$ to itself, which is applied to the $a$ values of the pixels. One can darken the image by using the square function. More precisely, we define the new value of a pixel of the image as $a^2/255$ (see figure \ref{fig-section4-square} in the center). Since the result is not generally an integer, it is rounded to the nearest integer. Similarly, for lightening the image, the value $a$ of each pixel is replaced by the rounding of $\sqrt{255 a} $. Figure \ref{fig-section4-square}, on the right, shows the obtained result. It will be noted that these two operations (square lightening and square root darkening) are inverse to one another.

% It will be noted that these two operations (square lightening and square root darkening) are inverse to one another.


\myfigure{
\imageTri{section4-negatif}
{section4-square}
{section4-sqrt}
{Negative}
{Square}
{Square root}
}{Changing the contrast.
}{fig-section4-square}


%%%%%%%%%%%%%%%%%%%%%%%%%%%%%%%%%%%%%%%%%%%%%%%%%% %%%%%%%%%%%%
\subsection{Manipulations of Color Contrast}

In order to manipulate the contrast of a color image, it is important to respect the color tones as much as possible. It is therefore simpler to manipulate only the luminance component $a = (r + v + b) / 3$, while maintaining the residue $(r-a, v-a, b-a)$ constant. For example, a change in contrast can be defined by raising the luminance $a$ to the $\ ga> 0$ in order to obtain
\[
	\tilde a = 255 \times \pa{ \frac{a}{255} }^{\ga} = 255 \times \text{exp}\pa{  \ga \times \text{ln}
		\pa{ \frac{a}{255} }  },
\]
(with Convention $\tilde a = 0$ when $a = 0$). It is noted that for $\ga = 1/2$ (respectively $\ga = 2$) the contrast change is found by squaring (respectively square root) introduced in the previous section. And of course, for $\ga = 1$, the luminance is unchanged.

% As $\ga$ is not necessarily an integer, it is important to use the exponential and the logarithm to define this change.

This change in contrast is then reflected on the color image by defining three channels $(\tilde r, \tilde v, \tilde b)$ of a new image by
\[
	\choice{
	\tilde r = \max(0, \min(255, r + \tilde a - a)),\\ 
	\tilde v = \max(0, \min(255, v + \tilde a - a)),\\
	\tilde b = \max(0, \min(255, b + \tilde a - a)).
	}	
\]
It is important to take the maximum with 0 and the minimum with 255 so that the result remains in the range $[0.255]$, and is displayed correctly. Figure \ref{fig-contrast-color} shows the result for different values of $\ga$. For $\ga <1$, the image looks clearer, while for $\ga> 1$, the image is darkened.

\newcommand{\imgsix}[1]{\includegraphics[width=0.16\linewidth]{#1}}

\myfigure{
\begin{tabular}{@{}c@{\hspace{1mm}}c@{\hspace{1mm}}c@{\hspace{1mm}}c@{\hspace{1mm}}c@{\hspace{1mm}}c@{}}
\imgsix{section4-constrast-1}&
\imgsix{section4-constrast-2}&
\imgsix{section4-constrast-3}&
\imgsix{section4-constrast-4}&
\imgsix{section4-constrast-5}&
\imgsix{section4-constrast-6}\\
$\ga=0,5$& 
$\ga=0,75$&
$\ga=1$&
$\ga=1,5$&
$\ga=2$&
$\ga=3$
\end{tabular}
}{Changing the contrast of a color image.
}{fig-contrast-color}





%%%%%%%%%%%%%%%%%%%%%%%%%%%%%%%%%%%%%%%%%%%%%%%%%%%%%%%%%%%%%%
%%%%%%%%%%%%%%%%%%%%%%%%%%%%%%%%%%%%%%%%%%%%%%%%%%%%%%%%%%%%%%
%%%%%%%%%%%%%%%%%%%%%%%%%%%%%%%%%%%%%%%%%%%%%%%%%%%%%%%%%%%%%%
\section{Images and Matrices}

%%%%%%%%%%%%%%%%%%%%%%%%%%%%%%%%%%%%%%%%%%%%%%%%%% %%%%%%%%%%%%
\subsection{Symmetry and Rotation}

An image is an array of numbers, with $n$ lines and $p$
columns. It is therefore easy to perform
some
\lien{https://en.wikipedia.org/wiki/Geometric_transformation}{geometric transformations}
on the image. The values of the pixels that make up this table (denoted $A$) can be
represented as $A = (a_{i,j})_{i,j}$
or index $i$ describes the set of numbers $\{1,\dots, n\}$
(the integers between 1 and n) and the index
$j$ the numbers $\{1,\dots, p\}$.
$a_{i,j}$ is said to be the value of the pixel at position $(i,j)$.

The array of pixels thus indexed is represented as
\[
A = 
\begin{pmatrix}
a_{1,1} &           &           &   & a_{1,p}\\
       &           &  \vdots   &   &  \\
	   &           & a_{i-1,j} &   & \\
\ldots & a_{i,j-1} & a_{i,j}   & a_{i,j+1} & \ldots\\
	   &           & a_{i+1,j} &   & \\
       &           &  \vdots   &   &  \\
a_{n,1} &           &           &   & a_{n,p}\\
\end{pmatrix},
\]
This corresponds to the representation of the image in the form of a matrix. Transposing this matrix corresponds to symmetry with respect to the main diagonal. This transposition is carried out on each of the three color components (see figure \ref{fig-geometric}, on the left).


\myfigure{
\imageTri
{section10-original}
{section10-transpose}
{section10-rotation}
{Matrice $A$}
{Matrix $B$ (transposed)}
{Matrix $C$ (rotation)}
}{Transpose and rotate.
}{fig-geometric}

It is also possible to carry out a rotation by
a quarter turn clockwise on the image. This is obtained by defining a matrix $C = (c_{i,j})_{j, i}$ of
$p$ lines and $n$
columns by $c_{j, i} = a_{n-i+1,j} $.
Figure \ref{fig-geometric}, right, shows the action of this rotation on an image.



%%%%%%%%%%%%%%%%%%%%%%%%%%%%%%%%%%%%%%%%%%%%%%%%%%%%%%%%%%%%%%
\subsection{Interpolation Between Two Images}

It is possible to carry out a
\lien{https://en.wikipedia.org/wiki/Dissolve_(filmmaking)}{transition between two images}
$A$ and $B$. It is therefore assumed that the two images have the same number $n$ of lines
and the same number $p$ of columns. $A = (a_{i,j})_{i,j}$ the pixels of image $A$ and
$B = (b_{i,j})_{i,j}$ the pixels of the image $B$.

For a value $t$ set between $0$ and $1$, the image
$C = (c_{i,j})_{i,j}$ is defined as
\[c_{i,j} = (1-t) a_{i,j} + t b_{i,j} .\]
It is the formula of a
\lien{https://en.wikipedia.org/wiki/Linear_interpolation}{linear interpolation}
between the two images. For a color image, this formula is applied to each of the
channels R, V and B.


It can be seen that for $t = 0$, the image $C$ is equal to the image
$A$. For $t = 1$, the image $C$ is equal to the image
$B$. When the value $t$ increases from 0 to 1, one thus obtains a fading
effect, since the image, which at first is close to image $A$
resembles more and more the image $B$. Figure \ref{fig-interp} shows the result obtained for 6 values of $t$ distributed between 0 and 1.

\myfigure{
\begin{tabular}{@{}c@{\hspace{1mm}}c@{\hspace{1mm}}c@{\hspace{1mm}}c@{\hspace{1mm}}c@{\hspace{1mm}}c@{}}
\imgsix{section11-interp-1}&
\imgsix{section11-interp-2}&
\imgsix{section11-interp-3}&
\imgsix{section11-interp-4}&
\imgsix{section11-interp-5}&
\imgsix{section11-interp-6}\\
Image $A$, t=0 & 
t=0.2 &
t=0.4 &
t=0.6 &
t=0.8 &
Image $B$, t=1
\end{tabular}
}{Linear interpolation.
}{fig-interp}



%%%%%%%%%%%%%%%%%%%%%%%%%%%%%%%%%%%%%%%%%%%%%%%%%% %%%%%%%%%%%%
%%%%%%%%%%%%%%%%%%%%%%%%%%%%%%%%%%%%%%%%%%%%%%%%%% %%%%%%%%%%%%
%%%%%%%%%%%%%%%%%%%%%%%%%%%%%%%%%%%%%%%%%%%%%%%%%% %%%%%%%%%%%%
\section*{Conclusion}

Mathematical processing of images is a very active field, where the theoretical advances are obtained using fast computational algorithms. These algorithms have important applications for the manipulation of digital contents. This article, however, only scratched the surface of the immense list of treatments that can be subjected to an image. We refer to the website \textit{A Numerical Tour of Signal Processing}\FootLink{http://www.numerical-tours.com/} for many more examples of image processing and links to other resources available online.


%%%%%%%%%%%%%%%%%%%%%%%%%%%%%%%%%%%%%%%%%%%%%%%%%%%%%%%%%%%%%%
%%%%%%%%%%%%%%%%%%%%%%%%%%%%%%%%%%%%%%%%%%%%%%%%%%%%%%%%%%%%%%
%%%%%%%%%%%%%%%%%%%%%%%%%%%%%%%%%%%%%%%%%%%%%%%%%%%%%%%%%%%%%%

\section*{Glossary}

\newcommand{\gloss}[1]{\item \textbf{#1}}

\begin{rs}
\gloss{random}: unpredictable value, such as noise disturbing images of bad qualities.
\gloss{Bit}: a basic unit for storing information in the form of 0 and 1 in a computer.
\gloss{Channel}: one of three elementary images that make up a color image.
\gloss{Edges}: the area of an image where the values of the pixels change rapidely, which corresponds to the contours of the objects that make up the image.
\gloss{Noise}: small perturbations that degrade the quality of an image.
\gloss{Square}:  the square $b$ of a value $a$ is $a \times a$. It is noted $a^2$.
\gloss{Contrast}: informal quantity that indicates how much difference there is between light and dark areas of an image.
\gloss{Image compression}: a method to reduce the amount of memory required to store an image on the hard disk.
\gloss{Binary coding}: writing of numeric values using only 0 and 1.
\gloss{Blur}: degradation of an image that makes the contours of objects unclear, and therefore difficult to locate precisely.
\gloss{Fade}: linear interpolation between two images.
\gloss{Color image}: a set of three grayscale images, which can be displayed on a color screen.
\gloss{Digital image}: an array of values that can be displayed on the screen by assigning a gray level to each value.
\gloss{Inverse}: operation that returns an image to its original state.
\gloss{JPEG-2000}: recent image compression method that uses a wavelet transform.
\gloss{Luminance}: average of the different channels in an image, which indicates the light output of the pixel.
\gloss{Matrix}: array of values, represented as $(a_{i,j})_{i,j}$.
\gloss{Median}: central value when sorting a set of values.
\gloss{Average}: the average of a set of values is their sum divided by their number.
\gloss{Grayscale}: grayscale used to display a digital image on the screen.
\gloss{integers}: numbers 0, 1, 2, 3, 4 ...
\gloss{Byte}: set of eight consecutive bits.
\gloss{Wavelets}: image transformation that is used by image compression JPEG-2000 method.
\gloss{Ascending order}: classifying a set of values from the smallest to the largest.
\gloss{Pixel}: a single element in the array of values that corresponds to a digital image.
\gloss{Quantization}: a method to reduce the set of possible values of a digital image.
\gloss{square root}: the square root $b$ of a positive value $a$ is the positive value $b$ such that $a = b \times b$. It is denoted $\sqrt{a}$.
\gloss{Resolution}: the size of an image (number of pixels).
\gloss{Exposed}: photograph of a scene too dark for which the photographic lens did not stay open long enough.
\gloss{Additive synthesis}: rule to construct any color from the three colors red, green and blue. This is the rule that governs the mixing of the colors of light beams when  illuminating a white wall.
\gloss{Subtractive synthesis}: rule to construct any color from the three cyan, magenta and yellow colors. This is the rule that governs the mixing of colors in paint.
\end{rs}
 

