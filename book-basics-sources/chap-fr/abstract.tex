% !TEX root = ../IntroImaging-FR.tex

\chapter*{Pr�sentation}

Les trois chapitres de ce texte sont ind�pendants et pr�sentent des introductions en douceur � quelques fondements math�matiques importants des sciences de l'imagerie :
\begin{itemize}
	\item Le chapitre~\ref{chap-shannon} pr�sente la th�orie de Shannon sur la compression et insiste en particulier sur l'entropie li�e au codage de l'information.
	\item Le chapitre~\ref{chap-images} pr�sente les bases du traitement d'images, en particulier des traitements importants (quantification, d�bruitage, couleurs).
	\item Le chapitre~\ref{chap-sparsity} pr�sente la th�orie de l'�chantillonnage, allant de l'�chantillonnage classique de Shannon � l'�chantillonnage comprim�. Il constitue �galement une introduction � la r�gularisation des probl�mes inverses.
\end{itemize}
Le niveau d'exposition pour les deux premiers chapitres est �l�mentaire. Le dernier chapitre pr�sente des concepts et r�sultats math�matiques plus avanc�s.

\tableofcontents